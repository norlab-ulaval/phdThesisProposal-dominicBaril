\section{Recherches futures}
\label{sec:recherches_futures}

Dans les \autoref{sec:papiers_soumis} ont été présentés le travail réalisé pour mettre au point le système de stations totales robotiques, ainsi que la collecte de données dans différents types d'environnement avec une précision sub-centimétrique.
Les deux premières problématiques de mon doctorat concernant la mise au point de ce système de collecte de données ainsi que l'obtention d'une calibration extrinsèque de précision ont été détaillées et résolues.
La présente section détaille les travaux de recherches que je planifie afin de répondre à ma dernière problématique de recherche, à savoir comment utiliser les données récoltées pour générer des vérités terrains précises dans l'ordre du millimètre, et effectuer des comparaisons de trajectoires avec ces vérités terrains.
Le calendrier détaillé de mes recherches restantes est présenté dans la \autoref{sec:calendrier}.

\subsection{Génération de vérités terrains}

La dernière partie de mon doctorat se concentrera donc sur la génération des vérités terrains, ce qui est la suite logique de mes travaux de recherches, ainsi que le but ultime de mon système de stations totales robotiques.
Cette dernière problématique se décompose en trois sous-parties:

\begin{enumerate}\bfseries
    \item La génération des vérités terrains,
    \item La gestion de l'incertitude,
    \item L'évaluation de trajectoires avec ces vérités terrains.
\end{enumerate}

La génération de vérités terrains, à savoir la pose de la plateforme robotique en six degrés de liberté, a déjà été succinctement évoquée dans le papier CRV 2021: sa faisabilité a été démontrée.
Celle-ci a été effectuée après l'interpolation des trajectoires de prismes par une minimisation point-à-point entre les positions théoriques des prismes les uns par rapport aux autres, et leurs positions mesurées sur le terrain.
Dans le papier d'ICRA 2023, nous avons relaté que l'utilisation de modules de filtrage permettait d'augmenter la précision de \SI{18}{\%} des résultats finaux.
Nous souhaitons appliquer le même principe à la génération des vérités terrains pour en augmenter la précision, et utiliser d'autres méthodes d'interpolation ou d'optimisation de trajectoires pour générer ces vérités terrains et évaluer quelles en sont les meilleures, notamment pour quantifier l'incertitude.

L'incertitude est rarement quantifiée lors de la génération de vérités terrains.
Elle l'est encore moins pour l'évaluation de trajectoires par rapport à celles-ci.
Avec notre système de trois stations totales et des trois prismes, nous avons la possibilité de pouvoir estimer l'incertitude grâce à la métrique de la distance inter-prisme.
C'est pourquoi l'objectif de mes prochaines recherches portera sur la manière de quantifier cette incertitude à l'aide des expériences effectuées.
Mon souhait est également d'exploiter cette incertitude pour ce qui concerne la comparaison de trajectoires.

L'évaluation de trajectoires avec des vérités terrains se fait depuis de nombreuses années.
La très grande majorité des évaluations utilise la norme euclidienne pour établir les comparaisons.
Cette norme est facile à mettre en application, mais elle ne prend pas en compte l'incertitude, ce qui peut fausser le résultat et donc rendre inutilisable la vérité terrain.
Ces situations peuvent se produire lorsqu'un \ac{GNSS} est utilisé pour générer une vérité terrain, alors que son signal n'est pas bon.
Une incertitude de plusieurs mètres peut alors apparaître comme ce fut le cas pour le jeu de données du Kitty Dataset \footnote{\url{https://www.cvlibs.net/datasets/kitti/eval_odometry.php}}, lequel compare des trajectoires données par des algorithmes de cartographie à sa vérité terrain issue d'un \ac{GNSS}.
En dessous de \SI{100}{m} d'évaluation par une métrique d'erreur de pose relative, les résultats peuvent être biaisés.
Les auteurs du jeu de données ont donc mis la distance d'évaluation pour une métrique d'erreur de pose relative à \SI{100}{m} au minimum pour éviter ces cas.
Dans mes recherches, je souhaite améliorer la prise en compte de l'incertitude concernant l'évaluation de trajectoires.
Cela permettra un résultat plus représentatif pour effectuer des comparaisons d'algorithmes entre eux.

Ces trois sous-parties de ma problématique finale seront développées dans un article de journal que je soumettrai fin mars 2023.
Le journal visé sera celui de ``Field Robotics'' \footnote{\url{https://www.journalfieldrobotics.org/Field_Robotics/Home.html}}.
Cet article de journal se concentrera sur la manière dont on générera les vérités terrains avec notre système de stations totales robotiques.
Il sera une suite logique des deux premiers articles, et présentera des jeux de données provenant de déploiements effectués durant une année entière (février 2022 à décembre 2022) afin de pouvoir montrer la plus-value de notre système dans différents types d'environnements et dans différentes conditions météorologiques.
Le détail des expériences à faire est présenté dans la prochaine sous-section.

\subsection{Expériences prévues}

Afin de préparer mes recherches futures, plus d'une dizaine de déploiements sont prévus en ces mois de novembre et décembre 2022.
Je collecterai les données manquantes à la présentation de mon travail dans l'article de journal.
Ces déploiements sont répartis en plusieurs étapes correspondant à différentes parties de ma recherche, à savoir l'évaluation de la précision du système de stations totales robotiques dans différentes conditions, la comparaison de celle-ci avec les \ac{GNSS} sur un plus grand nombre de données, et la récolte de plus de données en forêt de Montmorency.

Quelques évaluations ont déjà été réalisées et répertoriées dans l'article de CRV 2021 sur notre système afin de caractériser la précision de ce dernier.
Cependant, le jeu de données alors utilisé comprenait moins de \SI{500}{m} de trajectoires de la plateforme robotique et la méthode de calibration extrinsèque utilisée n'était pas aussi précise que celle développée dans l'article d'ICRA 2023.
Quelques expériences pour mieux comprendre notre système font défaut au jeu de données utilisé dans l'article d'ICRA 2023: à savoir la distance limite de la prise de données due à la distance maximale permise par les stations totales ainsi que la distance maximale des modules LoRa de communication.
Des expériences seront donc menées afin de quantifier ces limites, mais également afin de quantifier la précision en fonction de la distance de la prise de mesure. 
De plus, dans les articles de CRV 2021 et d'ICRA 2023 a été soulevée la question de l'influence de la position des prismes sur la plateforme robotique.
Des simulations et des expériences seront menées afin de pouvoir répondre à cette question.

Dans l'article de CRV 2021, a été effectuée une comparaison avec des données \ac{GNSS} en milieux ouverts ainsi qu'en forêt.
Pour faire suite aux récentes améliorations apportées à notre système, une collecte de données plus conséquente sera réalisée dans le but d'augmenter le nombre de données exploitables en plus de celles utilisées pour le papier d'ICRA 2023. 
Les données \ac{GNSS} ont pour le moment uniquement servi à comparer la précision du système par rapport à la métrique de la distance inter-prisme ou inter-\ac{GNSS}.
Les données que nous récolterons prochainement serviront cette fois à faire des comparaisons de trajectoires avec les vérités terrains.
De plus, les trajets effectués prendront en compte différents types de scénarios, tels que l'alternance de milieux ouverts ou couverts, afin de mettre en évidences les problèmes des \ac{GNSS} comparés à notre système de stations totales.

Enfin, il est prévu de faire des déploiements en forêt de Montmorency dans le but de recueillir des données avec météo variable: par temps de brouillard et de chutes de neige afin de quantifier la précision de notre système pendant ces épisodes climatiques.
De ce fait, au moins deux déploiements sont prévus à un mois d'écart: le premier au début du mois de novembre 2022 (sans neige), puis le second au début du mois de décembre (avec neige).
Les deux plateformes robotiques seront utilisées afin de maximiser la prise de données durant ces déploiements.
Ces expériences visant à comparer les jeux de données seront cependant toujours réalisées au même endroit, quelle que soit la météo.

Il est à noter que pour chacun des déploiements effectués, les données provenant de capteurs lidars seront également récoltées afin de pouvoir comparer la trajectoire donnée par notre algorithme de cartographie 3D à celle issue de notre système de stations totales robotiques ou de \ac{GNSS}.
Ainsi, grâce à chaque expérience menée, nous pourrons étudier un résultat spécifique et le présenter dans l'article de journal.
Le jeu de données de l'article d'ICRA 2023 sera également exploité pour certains résultats et certaines comparaisons, comme cela avait été prévu initialement.
Au total, nous souhaitons avoir plus de \SI{50}{km} de trajectoires de plateforme robotiques afin d'éliminer certaines valeurs aberrantes lors de la génération des résultats, et pouvoir ainsi accéder à un jeu de données complet prenant en compte différents types d'événements.
Cela nous est nécessaire si nous souhaitons trouver les limites de notre système, comme ce fus le cas par exemple dans l'article d'ICRA 2023.