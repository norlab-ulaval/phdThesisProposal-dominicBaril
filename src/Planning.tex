\section{Calendrier des recherches futures}
\label{sec:calendrier}

Le calendrier détaillé de la planification des objectifs pour la rédaction du prochain article de journal est située dans la \autoref{sec:Annexe} en annexe.
Ce calendrier s'étend de novembre 2022 à mars 2023.
Les différentes étapes de mes recherches y sont présentées: expériences à faire sur le terrain, génération des vérités terrains, prise en compte de l'incertitude pour ces générations, évaluation de trajectoires, et enfin rédaction de l'article de journal.

Mon article de journal rédigé, je me concentrerai sur la rédaction de ma thèse.
La \autoref{fig:gantt} montre le calendrier de la planification des objectifs pour le dépôt final de la thèse.
Je souhaite réaliser une thèse par articles.
Elle en comportera trois que j'ai publiés ou soumis: CRV 2021, ICRA 2023 et Field Robotics 2023.
S'il est accepté, l'article d'ICRA 2023 sera présenté à la conférence à la fin du mois de mai 2023.
Pour ce qui est de l'article du journal Field Robotics, il sera probablement publié d'ici un an ou deux, soit en 2024 ou 2025, après les différentes rondes de révisions.

Je prévois deux mois de rédaction de thèse juste après la soumission de mon article de journal, soit en avril et mai 2023. 
Je ferai mon dépôt initial au mois de mai 2023.
Après une période de 6 à 8 semaines, je pourrai effectuer ma défense durant le mois de juillet 2023.
Si ma thèse est acceptée, il m'appartiendra de faire les corrections données par mon comité pour un dépôt final escompté au mois d'août 2023.

\setlength{\fboxsep}{9pt}
\begin{figure}[htbp]
  \centering
  \definecolor{navyblue}{RGB}{21,80,130}
\setganttlinklabel{f-s}{}

\begin{ganttchart}[
     %Specs 0.43 0.43 1
     y unit title=0.43cm,
     y unit chart=0.55cm,
     x unit=1.1cm,  %1.2cm
     vgrid={*{1}{draw=none},*{1}{black}},
     title height=1,   %1
     title label font=\bfseries\footnotesize,
     % bar 0.4 10pts
     bar/.style={fill=navyblue},
     bar height=0.4,
     bar label font=\footnotesize,
     bar label node/.append style={left=10pt},  %10
     % group 0 0.6 0.3 0.2 0.3 10pts
     group right shift=0,
     group top shift=0.6,
     group height=.3,
     group peaks width={0.2},
     group peaks height={0.3},
     group label node/.append style={left=10pt},
     group label font=\bfseries\footnotesize,
     % milestone 0.4 1 10pts
     milestone/.append style={xscale=0.4, yscale=1},
     milestone label node/.append style={left=10pt},
     milestone label font=\itshape\footnotesize
     ]{1}{8}
    \gantttitle[]{2024}{8}
    \\              
    \gantttitle{J}{1} \gantttitle{F}{1} \gantttitle{M}{1} 
    \gantttitle{A}{1} \gantttitle{M}{1} \gantttitle{J}{1} 
    \gantttitle{J}{1} \gantttitle{A}{1} 
    \\
    %--------------------------------------       
    \ganttgroup{Thesis writing}{1}{3}\\
    
    \ganttbar{Articles insertion}{1}{1}
    \ganttbar[inline, bar label font/.append=\color{white}]{}{1}{1}\\
    
    \ganttbar{Final results chapter}{2}{2}
    \ganttbar[inline, bar label font/.append=\color{white}]{}{2}{2}\\
    
    \ganttbar{Introduction, conclusion}{3}{3}
    \ganttbar[inline, bar label font/.append=\color{white}]{}{3}{3}\\
	
	\ganttmilestone{Thesis submitted}{3}\\
	
    %--------------------------------------       
    \ganttgroup{Final experiments}{1}{2} \\ 
    
    \ganttbar{DRIVE impact on UGV path following}{1}{2}
    \ganttbar[inline, bar label font/.append=\color{white}]{}{1}{2}\\
    
    %--------------------------------------       
    \ganttgroup{\emph{ICRA 2024}}{2}{5} \\ 
    
    \ganttbar{Paper corrections}{2}{2}
    \ganttbar[inline, bar label font/.append=\color{white}]{}{2}{2}\\
    
    \ganttmilestone{Re-submission}{2}\\
    
    \ganttbar{Presentation preparation}{5}{5}
    \ganttbar[inline, bar label font/.append=\color{white}]{}{5}{5}\\
    
    \ganttmilestone{Article presentation}{5}\\
    
    %--------------------------------------       
    \ganttgroup{Thesis defense}{4}{8}\\
    
    \ganttbar{Thesis review}{4}{6}
    \ganttbar[inline, bar label font/.append=\color{white}]{}{4}{6}\\
    
    \ganttbar{Defense preparation}{4}{6}
    \ganttbar[inline, bar label font/.append=\color{white}]{}{4}{6}\\
    
    \ganttmilestone{Defense}{6}\\
    
    \ganttbar{Final reviews}{7}{8}
    \ganttbar[inline, bar label font/.append=\color{white}]{}{7}{8}\\
    
    \ganttmilestone{Final submission}{8}\\
    
\end{ganttchart}

  \caption{Diagramme de Gantt contenant les objectifs devant être accompli d'ici au dépôt final de la thèse.
  }
  \label{fig:gantt}
\end{figure}
\setlength{\fboxsep}{12pt}

