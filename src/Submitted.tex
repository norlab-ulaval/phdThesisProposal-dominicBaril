\section{Current scientific production}
\label{sec:submitted}

This section describes the current scientific done through this Ph.D. thesis work and collaborations done with other researchers. 
First, the three articles conducted directly for which I acted as first author.
All of these scientific contributions are related to the subproblems states in~\autoref{sec:introduction}.
Then all of the worth in which I have participated as co-author is described briefly.
Since mobile robotics is a field requiring various expertise and human ressources to conduct field deployments, all scientific production presented includes multiple co-authors.

%\subsection{Format de soumission des papiers avec co-auteurs}

%Le champ de la robotique mobile est un domaine qui exige une expertise dans plusieurs domaines.
%C'est pourquoi mes publications ont été faites en collaboration avec plusieurs co-auteurs œuvrant dans des domaines connexes.
%Typiquement, le premier auteur est la personne dirigeant les efforts et tâches à effectuer pour la publication du papier (i.e. typiquement un étudiant faisant de la recherche), tandis que les derniers co-auteurs sont ceux conseillant sur la stratégie de publications ainsi que sur la vision à long terme du projet (i.e. typiquement un professeur ou chercheur supervisant le travail de l'étudiant).
%Quant aux autres co-auteurs, ils aident à effectuer plusieurs parties du travail de recherches, que ce soit pour aider à faire les expériences sur le terrain, pour traiter les résultats, pour finaliser l'écriture de la théorie ou pour aider à la rédaction.
%Mon rôle en tant que premier auteur fut de définir la problématique que les papiers aborderont, de travailler sur la théorie et la partie expérimentale avec le traitement des données, et enfin la supervision de la rédaction du papier.

\subsection{Articles published and submitted as first author}
\begin{center}
	\textbf{\fullcite{Baril2020}}
\end{center}
The first article that I have published was submitted to the ``Conference on Robot and Vision (CRV)'', in May 2020.
This article aims to evaluate the performance of~\acp{SSMR} kinematic motion models on dry concrete and snow-covered terrain.
For this article
In this article, we collected a total of~\SI{2}{\kilo\meter} of human driving data to evaluate four kinematic models from the literature.
We leverage lidar point cloud registration based on the~\ac{ICP} algorithm to generate ground truth localization.
The resulting contributions are as follows:
\begin{enumerate}
	\item validate their fitness for a heavier platform on a relatively uniform concrete terrain;
	\item evaluate their performance for snow-covered terrain using more than \SI{2}{\kilo\meter} of trajectories traveled; and
	\item highlight the impact of angular motion on the accuracy of \acp{SSMR} kinematic modeling.
\end{enumerate}

The four kinematic models evaluated are the extended differential-drive asymmetrical, the extended differential-drive symmetrical~\citep{Mandow2007}, the full linear~\citep{Anousaki2004} and~\ac{ROC}-based~\citep{Wang2015}.
We also show that models with less parameters tend to perform better for angular prediction and models with more parameters perform better for translation prediction, due to their ability to predict non-zero lateral motion.
However, once trained, the performance of all models is similar for both terrain types, suggesting that all kinematic models evaluated behave similarly.
The largest prediction error occurs when the vehicle's angular velocity is at its maximum, which leads to the highest amount of vehicle slip.

Additionally, training kinematic models with empirical driving data leads to significant prediction error reduction, for both concrete and snow-covered terrain.
The relation between training window and prediction error is also studied in this work, clearly showing that models perform best when predicting for the same horizon for which they were trained.
We show that for the same commanded angular velocity, angular velocity is higher on snow-covered terrain than concrete. 
This phenomenon is due to the high friction caused by the tire deformation occurring during skidding on concrete, compared to soft terrain deformation on snow-covered terrain.

The take home message for this published paper was that kinematic motion models are adequate for predicting~\ac{SSMR} motion, both on dry concrete and snow-covered terrain, however they require a training dataset dependent to vehicle and terrain properties.
During the experimental work conducted for this paper, we imitated similar work by having a human operator to stimulate as many commands as possible, however this process lead to biased command stimulation and forward-only driving and proved to be time-consuming.
Since deploying~\acp{UGV} in off-road environments is a complex endeavor, reducing the time requiring to generate a motion model that is accurate enough for stable autonomous navigation is key.

\begin{center}
	\textbf{\fullcite{Baril2023}}
\end{center}

Le second article sur lequel j'ai travaillé a été soumis à la conférence appelée ``International Conference on Robotics and Automation (ICRA)'', en septembre 2023.
Cet article se concentre sur les différentes méthodes de calibration extrinsèque employées pour exprimer les données provenant de plusieurs stations totales robotiques dans un référentiel commun, et répond donc à la deuxième problématique de mes travaux de recherches.
Dans cet article, nous proposons une nouvelle méthode dynamique qui se base sur les trois prismes suivis par nos trois stations totales robotiques.
Une comparaison est effectuée entre toutes les méthodes étudiées: la nôtre augmente la précision de \SI{25}{\%} par rapport à la meilleure des méthodes utilisées dans l'état de l'art.
Pour cet article, j'ai rédigé une bonne partie du texte, développé une partie du code pour le traitement des données, fait les expériences sur le terrain, et interprété les résultats obtenus.
La citation complète de ma publication est la suivante:

L'article décrit dans un premier temps l'état de l'art à propos des différentes méthodes de calibration extrinsèque existantes en géomatique et arpentage.
La première méthode est appelée calibration en deux-points.
Elle a uniquement besoin de deux points statiques dont on connaît très précisément la position relative, au moins au millimètre près.
Les positions sont mesurées par chacune des stations totales et la méthode nous donne les transformations rigides ente les référentiels respectifs.
La seconde méthode, qui est la plus utilisée pour faire une calibration extrinsèque, est celle des points statiques de calibration.
Cette méthode consiste à prendre les positions statiques de nombreux points de contrôles, à recaler leurs positions, ce qui nous donne les transformations rigides entre les référentiels respectifs des stations totales.
La troisième méthode est identique à celle précédemment citée, mais la position des points de contrôle est prise de manière dynamique avec un prisme en mouvement. 
Toutes ces méthodes, en plus de la nouvelle que nous proposons, sont décrites mathématiquement dans la section théorie du papier.
L'état de l'art du papier donne également les principales sources de bruits à prendre en compte lors d'expérimentations avec plusieurs stations totales robotiques.

La nouvelle méthode de calibration extrinsèque que nous proposons se base sur les distances inter-prismes positionnées sur la plateforme robotique.
Les distances pouvant être déterminées au millimètre près avec une station totale, nous pouvons minimiser la distance de ces distances inter-prismes avec une fonction de coût optimisée dans l'algèbre de Lie par une méthode des moindres carrés. 
Le résultat de cette minimisation nous donne directement les transformations rigides entre les référentiels des différentes stations totales robotiques.
Pour pouvoir appliquer cette minimisation, nous pré-traitons les données brutes provenant des stations totales avec un pipeline dédié filtrant les données dans un premier temps. 
Puis, dans un deuxième temps, celles-ci sont interpolées permettant les mesures des stations totales à un même instant.
Ceci n'est pas le cas habituellement puisque les données sont prises de manière asynchrone.

Afin de comparer les différentes méthodes, un jeu de données totalisant plus de \SI{30}{km} de trajectoire de la plateforme robotique a pu être enregistré lors de 40 expériences différentes: elles ont été réalisées entre fin février 2022 et septembre 2022 lors de 15 déploiements.
Ces déploiements se déroulèrent dans deux types d'environnement: à l'extérieur, tout d'abord sur le \underline{campus de l'Université Laval} où se situent des bases de calibrations statiques permettant de réaliser la calibration en deux-points, puis dans les \underline{tunnels de l'Université Laval}, longs et droits sur plusieurs centaines de mètres à certains endroits.
Le choix s'est porté sur ces lieux du fait de leurs différentes configurations spatiales et des expériences de cartographie 3D avec lidars qu'y mène le Norlab.
Comme suite à chaque déploiement, une station totale a mesuré la position des prismes et de certains capteurs avec précision afin que notre nouvelle méthode de calibration extrinsèque puisse être appliquée.

Grâce à ce jeu de données, des tests de sensibilités et d'ablations ont pu être effectués sur les différents paramètres et modules du pipeline de pré-traitement.
Il a été démontré que le pipeline augmente la précision des résultats de \SI{18}{\%} grâce au filtrage des données erronées.
De plus, une simple interpolation linéaire des données est suffisante en comparaison d'une utilisation de ``Gaussian Process'' pour laquelle une meilleure précision était attendue.
La comparaison entre les différentes méthodes de calibration extrinsèque nous démontre que la nôtre est la plus précise.
Son exactitude est cependant plus faible de quelques millimètres en comparaison de la méthode des points de contrôles statiques.
Globalement, toutes les méthodes sont plus précises que l'utilisation de \ac{GNSS}.

Dans cet article, nous avons proposé une nouvelle méthode de calibration extrinsèque qui ne se base pas sur la mesure de points statiques, ce qui permet d'augmenter le temps consacré aux expériences lors de déploiements.
Cette nouvelle méthode s'accompagne d'un pipeline de pré-traitement des données venant des stations totales robotiques. 
L'utilisation du pipeline augmente de \SI{18}{\%} la précision des résultats, et ceux-ci sont plus précis de \SI{25}{\%} car combinés à notre nouvelle méthode de calibration extrinsèque.
Notre nouvelle méthode de calibration est limitée dans certains types d'environnements comme les tunnels, où les trajectoires sont longues et droites, ce qui empêche notre minimisation de converger vers le résultat attendu.
Des simulations plus poussées seront effectuées afin de mieux caractériser ces limites d'utilisation.
La position des prismes sur la plateforme robotique a également un effet sur le résultat, qui doit être davantage quantifié avec précision. 
Cet article met à notre disposition tous les moyens de recueillir des données précises permettant de reconstruire les vérités terrains précises qui seront utiles pour l'évaluation des algorithmes de cartographie 3D.
Le code ainsi que le jeu de données utilisés sont disponibles en ligne \footnote{\url{https://github.com/norlab-ulaval/RTS_Extrinsic_Calibration}}.

\subsection{Articles published and submitted as co-author}

\textbf{\fullcite{Vlad2022}}: 
Ce papier présente un moyen d'améliorer la précision de la localisation par lidar grâce à la prise en compte du vecteur de gravité donné par une centrale inertielle. 
Pour ce papier, j'ai effectué les comparaisons de trajectoire entre celles données par notre algorithme de cartographie amélioré et celles des vérités terrains prises par des \ac{GNSS}. 
J'ai également apporté mon aide à la prise de jeux de données sur le terrain et à la rédaction.
\\

\textbf{\fullcite{Baril2022}}:
Cet article de journal présente un pipeline de ``teach-and-repeat'' qui a été testé en forêt Montmorency. 
Avec ce pipeline, la plateforme robotique a pu effectuer près de \SI{20}{km} de trajectoire en toute autonomie, dans des conditions boréales. 
Pour ce faire, j'ai enregistré sur le terrain des données de vérités terrains à l'aide de mon système de stations totales et de \ac{GNSS}, puis j'ai traité les données recueillies. 
J'ai également participé à la rédaction de l'article.
\\

\textbf{\fullcite{Roucek2021}}:
En février 2020, j'ai participé à la compétition du DARPA (``Defense Advanced Research Projects Agency'') Urban Challenge. 
Ce papier présente le système utilisé lors de la compétition, ainsi que les résultats obtenus. 
J'ai participé à la compétition, traité les données en lien avec la cartographie 3D et effectué l'évaluation de la précision de notre algorithme pour la localisation des plateformes utilisées.
\\

\textbf{\fullcite{Chahine2021}}: 
Ce papier présente un algorithme de cartographie 3D se basant sur différents capteurs visant ainsi à améliorer la précision du résultat (lidar, caméra et \ac{GNSS} ensemble).

C'est dans ce papier que j'ai présenté pour la première fois le début de mes recherches avec mon système de stations totales. 
Mais seule une station totale a pu alors être utilisée du fait du manque d'espace pour les cibles sur le sac à dos avec les capteurs.
J'ai aidé à la prise de données sur le terrain, ainsi qu'à leur traitement pour l'obtention de la vérité terrain, et j'ai également participé à la rédaction de l'article.
\\

\textbf{\fullcite{Baril2020}}:
Cet article présente l'évaluation de modèle cinématique sur différents types de terrains: asphalte ou neige. 
Le but de l'article est de présenter les résultats des différents modèles et de discuter de leur efficacité. 
Pour cet article, j'ai pris des données sur le terrain et j'ai traité celles provenant de notre algorithme de cartographie. 
J'ai également aidé à la rédaction.
\\

\textbf{\fullcite{Laconte2021}}:
Ce papier présente l'état de l'art pour les différentes méthodes de localisation de véhicules sur des autoroutes, principalement par la détection des lignes via des systèmes de caméras et de lidars.
J'ai aidé à la rédaction de l'article. Aucune donnée n'a été prise sur le terrain.
\\

\textbf{\fullcite{Vaidis2020}}:
Cet article est un article de ``workshop'' pour la conférence ICRA 2021, en ce sens non pertinent pour ma thèse.
Il aborde l'amélioration de notre algorithme de cartographie avec des contraintes de rotation fixées dans l'algèbre de Lie via des données issues d'une centrale inertielle.
Pour ce papier, j'ai effectué les expériences, traité les données et rédigé la grande partie de l'article.